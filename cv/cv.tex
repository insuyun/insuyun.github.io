% Last updated: 5/21/2017

\documentclass[11pt,letterpaper]{article}

\usepackage{enumitem}
\usepackage{parskip}
\usepackage{url}
\usepackage{hyperref}
\usepackage[left=0.75in,top=0.6in,right=0.75in,bottom=0.6in]{geometry}
\usepackage{bold-extra}
\usepackage{ifthen}
\usepackage{relsize}
\usepackage{comment}
\usepackage{textcomp}
\usepackage{lmodern}
\usepackage{etaremune}

\newcommand\email[1]{\href{mailto:#1}{#1}}
\newcommand{\sectiontitle}[1]{\vspace{1em}\textbf{\Large \textsc{#1}}\vspace{0.5em}\hrule}
\newcommand{\subsectiontitle}[3][]{{\bf #2}\ifthenelse{\equal{#1}{}}{}{, #1}
\hfill {#3}}

\newenvironment{topic}[3][]
{
  \subsectiontitle[#1]{#2}{#3}\vspace{-0.5em}
	\begin{itemize}[leftmargin=1.5em]
    \small
		\renewcommand\labelitemi{}
		\itemsep=-0.4em
}
{
	\end{itemize}
}

\newcommand{\entry}[2]{\item {#1 \hfill {#2} \vspace{0em}}}

\newenvironment{subtopic}
{
  \begin{itemize}[label=$\bullet$]
  \vspace{-5px}
}
{
  \end{itemize}
}

\newcommand{\cc}[1]{\mbox{\smaller[0.5]\texttt{#1}}}


\begin{document}

% Name, phone, email, address
{\bf\huge Insu Yun} \vspace{1em}\\
\noindent\begin{tabular}[t]{@{}l}
  Assistant Professor \\
  School of Electrical Engineering, \\
  Korea Advanced Institute of Science and Technology (KAIST)
\end{tabular}
\hfill
\begin{tabular}[t]{r@{}}
\\
Email: \email{insuyun@kaist.ac.kr} \\
Web: \href{https://insuyun.github.io}{https://insuyun.github.io} \\
\end{tabular}

%
% Research Interests
%

\sectiontitle{Research Interests}
Binary analysis, system security and applied cryptography.

%
% Education
%

\sectiontitle{Education}
\begin{topic}{Georgia Institute of Technology}{Aug. 2015 -- Dec. 2020}
	\item Ph.D. in Computer Science
	\item Advisor: Dr. Taesoo Kim
\end{topic}

\begin{topic}{Korea Advanced Institute of Science and Technology (KAIST)}{Sep. 2008 -- Feb. 2015}
\item B.S. in Computer Science \& Mathematics
\end{topic}

%
% Publications
%

\sectiontitle{Publications}

\textbf{International Conferences}
\begin{etaremune}
  {{ INTL_CONF }}
\end{etaremune}

\textbf{International Journal}
\begin{etaremune}
  {{ INTL_JOURNAL }}
\end{etaremune}

\textbf{Domestic Conferences}
\begin{etaremune}
  {{ DOM_CONF }}
\end{etaremune}

\textbf{Thesis}
\begin{etaremune}
\item \textbf{Concolic Execution Tailored for Hybrid Fuzzing}  \\
{\footnotesize
  \textbf{Insu Yun} \\
  Ph.D. thesis, Georgia Institute of Technology \\
Atlanta, GA, December 2020
}
\end{etaremune}


%
% Work Experience
%

\sectiontitle{Work Experience}
\begin{topic}[Daejeon, South Korea]{KAIST}{Feb. 2021 --}
\item Assistant Professor
\end{topic}
\begin{topic}[Research Intern, Seattle, WA]{Microsoft Research}{May. 2017 -- Aug. 2017}
\item Contributed to REPT, a system that utilizes Intel Processor Trace to diagnose production failures
  \item Mentor:  Weidong Cui
\end{topic}
\subsectiontitle[Research Assistant, Atlanta, GA]{Georgia Tech}{Aug. 2015 -- Dec. 2020} \\
\begin{topic}[Software Developer, Seoul, Korea]{Korean Cyber Command}{Apr. 2012 -- Jan. 2014}
  \item Served for the mandatory military service
\end{topic}

\sectiontitle{Professional Activities}
\begin{topic}{International Conference (Committee)}{}
  \item{Program Committee, \emph{Network and Distributed System Security Symposium (NDSS)}, 2024}
  \item{Program Committee, \emph{IEEE Symposium on Security and Privacy (Oakland)}, 2024}
  \item{Program Committee, \emph{ACM Conference on Security and Privacy in Wireless and Mobile Networks (WiSec)}, 2023}
  \item{Artifact Evaluation Committee, \emph{USENIX Security Symposium (Security)}, 2023}
  \item{Program Committee, \emph{ACM Conference on Security and Privacy in Wireless and Mobile Networks (WiSec)}, 2022}
  \item{Organization Committee, \emph{ACM Conference on Computer and Communications Security (CCS)}, 2021}
\end{topic}

\begin{topic}{Domestic Conference (Committee)}{}
  \item{Organization Committee, \emph{Conference on Information Security and Cryptography Summer (CISC-S)}, 2021}
\end{topic}

% \begin{topic}{International Conference (External Reviewer)}{}
%   \item{External Reviewer, \emph{Network and Distributed System Security Symposium (NDSS)}, 2023}
%   \item{External Reviewer, \emph{Network and Distributed System Security Symposium (NDSS)}, 2022}
%   \item{External Reviewer, \emph{Network and Distributed System Security Symposium (NDSS)}, 2021}
%   \item{External Reviewer, \emph{Network and Distributed System Security Symposium (NDSS)}, 2020}
%   \item{External Reviewer, \emph{ACM Conference on Computer and Communications Security (CCS)}, 2019}
%   \item{External Reviewer, \emph{USENIX Symposium on Operating Systems Design and Implementation (OSDI)}, 2018}
%   \item{External Reviewer, \emph{USENIX Security Symposium (Security)}, 2018}
%   \item{External Reviewer, \emph{USENIX Annual Technical Conference (ATC)}, 2018}
%   \item{External Reviewer, \emph{Network and Distributed System Security Symposium (NDSS)}, 2018}
%   \item{External Reviewer, \emph{ACM Conference on Computer and Communications Security (CCS)}, 2017}
%   \item{External Reviewer, \emph{ACM Conference on Computer and Communications Security (CCS)}, 2015}
%   \item{External Reviewer, \emph{IEEE Symposium on Security and Privacy (Oakland)}, 2015}
%   \item{External Reviewer, \emph{ACM Conference on Computer and Communications Security (CCS)}, 2014}
%   \item{External Reviewer, \emph{ACM Conference on Security and Privacy in Wireless and Mobile Networks (WiSec)}, 2014}
%   \item{External Reviewer, \emph{IEEE Symposium on Security and Privacy (Oakland)}, 2014}
%   \item{External Reviewer, \emph{Network and Distributed System Security Symposium (NDSS)}, 2014}
%   \item{External Reviewer, \emph{IEEE Symposium on Security and Privacy (Oakland)}, 2013}
% \end{topic}

% Teaching Experience
\sectiontitle{Teaching Experience}
% \textbf{Instructor}
\begin{topic}{}{}
  \entry{Programming Structures for Electronical Engineering (EE209 at KAIST)}{Fall 2022}
  \begin{subtopic}
    \item Evaluation --  Average: 4.65 / 5
  \end{subtopic}
  \entry{Software development environment and tools practice (EE485-A at KAIST)}{Fall 2022}
  \begin{subtopic}
    \item Evaluation --  Average: 4.43 / 5
  \end{subtopic}
  \entry{My Life and Career in EE II (EE485-C at KAIST)}{Fall 2022}
  \begin{subtopic}
    \item Evaluation --  Average: 4.70 / 5
  \end{subtopic}
  \entry{Software Security (EE595-B at KAIST)}{Spring 2022}
  \begin{subtopic}
    \item Evaluation --  Average: 5 / 5
  \end{subtopic}
  \entry{My Life and Career in EE I (EE485-C at KAIST)}{Spring 2022}
  \begin{subtopic}
    \item Evaluation --  Average: 4.65 / 5
  \end{subtopic}
  \entry{Programming Structures for Electronical Engineering (EE209 at KAIST)}{Fall 2021}
  \begin{subtopic}
    \item Evaluation --  Average: 4.34 / 5
  \end{subtopic}
  \entry{Software development environment and tools practice (EE485-A at KAIST)}{Fall 2021}
  \begin{subtopic}
    \item Evaluation --  Average: 4.34 / 5
  \end{subtopic}
  \entry{My Life and Career in EE II (EE485-C at KAIST)}{Fall 2021}
  \begin{subtopic}
    \item Evaluation --  Average: 4.57 / 5
  \end{subtopic}
  \entry{Software Security (EE595-B at KAIST)}{Spring 2021}
  \begin{subtopic}
    \item Evaluation --  Average: 4.9 / 5
  \end{subtopic}
\end{topic}

% \textbf{Teaching Assistant}
% \begin{topic}{}{}
%   \entry{Teaching Assistant, Information Security Lab -- Official (CS8803 at Georgia Tech)}{Fall 2018}
%   \begin{subtopic}
%     \item Evaluation --  Overall Effectiveness: 5 / 5
%   \end{subtopic}
%   \entry{Teaching Assistant, Information Security Lab -- Unofficial (CS8803 at Georgia Tech)}{Fall 2017}
%   \entry{Teaching Assistant, Information Security Lab -- Official (CS6265 at Georgia Tech)}{Fall 2016}
%   \begin{subtopic}
%     \item Evaluation --  Overall Effectiveness: 4.9 / 5
%   \end{subtopic}
%   \entry{Teaching Assistant, Information Security Lab -- Unofficial (CS6265 at Georgia Tech)}{Fall 2015}
%   \entry{Head Instructor, Information Security class for freshmen (KAIST)}{Mar. 2009 -- Aug. 2011}
% \end{topic}

% Honors & Awards
\sectiontitle{Honors \& Awards}
\begin{topic}{Academic awards}{}
  \entry{Best Lecture Award, KAIST Electrical Engineering}{Sep. 2021}
  \entry{Jay Lepreau Best Paper Award, USENIX OSDI 2018}{Aug. 2018}
  \entry{Distinguished Paper Award, USENIX Security 2018}{Aug. 2018}
\end{topic}

\begin{topic}{Capture-the-flag(CTF) contests}{}
  \entry{DEFCON 26 CTF, 1st place (Team DEFKOR00T)}{Aug. 2018}
  \entry{DEFCON 24 CTF, 3rd place (Team DEFKOR)}{Aug. 2016}
  \entry{DARPA Cyber Grand Challenge (Team Disekt)}{Aug. 2016}
  \entry{DEFCON 23 CTF, 1st place (Team DEFKOR)}{Aug. 2015}
  \entry{Whitehat contest 2014 (Team SysSec)}{Nov. 2014}
  \entry{DEFCON 22 CTF, 10th place (Team GoN)}{Aug. 2014}
  \entry{SECCON CTF 2014, 1st place (TOEFL Beginner)}{Feb. 2014}
  \entry{Codegate CTF 2012, 3rd place (Team GoN)}{Apr. 2012}
  \entry{Secuinside CTF, 3rd place (Team GoN)}{Oct. 2011}
  \entry{ISEC CTF, 1st place (Team GoN)}{Sep. 2011}
  \entry{DEFCON 18 CTF, 3rd place (Team GoN)}{Aug. 2010}
  \entry{Codegate CTF 2010, 5th place (Team GoN)}{Apr. 2010}
  \entry{KISA HDCON, Gold Medal, 2nd place (Team GoN)}{May 2009}
  \entry{Codegate CTF 2009, 4th place (Team GoN)}{Apr. 2009}
\end{topic}
%
\begin{topic}{Bug Bounty (by me)}{}
  \entry{PSV-2021-0304: afpd auth bypass (\$300), NETGEAR}{Mar. 2021}
  \entry{Pwn2Own Apple Safari with a kernel privilege escalation (\$70K), Zero Day Initiative}{Mar. 2020}
  \entry{Apple Safari sandbox escape (\$20K), Apple}{Dec. 2019}
  \entry{Three integer overflow vulnerabilities in PHP (\$1,500), the Internet Bug Bounty}{Jun. 2016}
  \entry{An Integer Overflow in Python zipimport (\$1,000), the Internet Bug Bounty}{Apr. 2016}
\end{topic}

% TODO: Add type confusion
\begin{topic}{Bug Bounty (by students)}{}
  \entry{Type confusion in V8 (\$7,000) by Haein Lee, Google}{Mar. 2023}
  \entry{LTE authentication bypass in Exynos (\$14,760) by Eunsoo Kim and CheolJun Park, Samsung}{Feb. 2022}
\end{topic}
%
\begin{topic}{Scholarships}{}
  \entry{National Research Foundation of Korea Scholarship for Undergraduate}{Mar. 2008 -- Dec. 2013}
\end{topic}

\begin{comment}
\sectiontitle{Reported Security Vulnerabilities}
  {{ CVE }}
\end{comment}

% Invited Talks
\sectiontitle{Invited Talks}

\begin{topic}{How to build Skynet --- a system that hacks systems}{}
  \entry{Keynote speech at TyphoonCon}{Jun. 2023} % 2023.06.15
\end{topic}

\begin{topic}{Human-friendly binary analysis}{}
  \entry{Presented at Korea Computer Congress (KCC)}{Jun. 2023} % 2023.06.19
\end{topic}

\begin{topic}{Exploit in the wild}{}
  \entry{Presented at ETRI}{Jun. 2023} % 2023.06.14
\end{topic}

\begin{topic}{Hacking 101}{}
  \entry{Presented at WISC}{Sep. 2022} % 2022.9.6
\end{topic}

\begin{topic}{Attack and Defenses for Heap Vulnerabilities in 2022}{}
  \entry{Seminar at ETRI}{Apr. 2022} % 2022.4.12
\end{topic}

\begin{topic}{Comparative Analysis of Baseband Software and Cellular Specifications for Finding Vulnerabilities}{}
  \entry{Seminar at UNIST}{May. 2023} % 2023.5.17
  \entry{Seminar at Security@KAIST}{Jun. 2022} % 2022.06.22
  \entry{Seminar at Cyber Operations Command}{Jun. 2022} % 2022.06.10
\end{topic}

\begin{topic}{Memory Allocator Security}{}
  \entry{Presented at Computer System Society Conference (CSC)}{Feb. 2023} % 2023.02.07
  \entry{Seminar at UNIST}{May. 2022} % 2022.5.18
  \entry{Seminar at Yonsei university}{Apr. 2022} % 2022.4.28
  \entry{Seminar at Sungkyunkwan university}{Apr. 2022} % 2022.4.27
  \entry{Seminar at ETRI}{Jan. 2022} % 2022.1.25
  \entry{Seminar at National Security Research Institute (NSRI)}{Dec. 2021} % 2021.12.09
  \entry{Seminar at Securty@KAIST}{Nov. 2021} % 2021.11.24
  \entry{Seminar at KAIST GSIS}{Nov. 2021} % 2021.11.23
\end{topic}

\begin{topic}{Browser Security: Hacking \& Research}{}
  \entry{Presented at Open Theori Research Seminar \#6}{Dec. 2021} % 2021.12.23
  \entry{Seminar at Hanyang University}{Nov. 2021} % 2021.11.17
  \entry{Presented at KR Becks Meetup \#1 by LINE}{Aug. 2021} % 2021.08.13
  \entry{Seminar at Security@KAIST}{Jun. 2021} % 2021.06.23
\end{topic}

\begin{topic}{HardsHeap: A Universal and Extensible Framework for Evaluating Secure Allocators}{}
  \entry{Presented at ACM CCS 2021}{Nov. 2021}
\end{topic}

\begin{topic}{Automatic Techniques to Systematically Discover New Heap Exploitation Primitives}{}
  \entry{Presented at USENIX Security 2020}{Aug. 2020}
\end{topic}

\begin{topic}{Scalable and Automatic Vulnerability Discovery Beyond Random Testing}{}
  \entry{Seminar at Seoul National Univeristy}{Mar. 2019} % 2019.03.11
\end{topic}

\begin{topic}{QSYM: A Practical Concolic Execution Engine Tailored for Hybrid Fuzzing}{}
  \entry{Presented at USENIX Security 2018}{Aug. 2018}
\end{topic}

\begin{topic}{APISan: Sanitizing API Usages through Semantic Cross-checking}{}
  \entry{Presented at USENIX Security 2016}{Aug. 2016}
\end{topic}

\sectiontitle{Advising and Mentoring}
\begin{itemize}
  \item \textbf{Ph.D./M.S Students}
    \begin{topic}{}{}
      \entry{Haein Lee}{Starting from Spring 2022}
    \end{topic}

  \item \textbf{M.S. Students}
    \begin{topic}{}{}
      \entry{Minwoo Baek}{Starting from Spring 2022}
      \entry{Wonyeong Jung}{Starting from Spring 2022}
      \entry{Junyeong Park}{Starting from Spring 2022}
      \entry{Dongok Kim}{Starting from Spring 2023}
    \end{topic}

  \item \textbf{Alumni}
    \begin{topic}{}{}
      \entry{Hyunsik Jeong (Co-advising with Yongdae Kim)}{M.S. in Fall 2021} \\
      % TODO: Publication
      First employment: S2W

      \entry{Hyunseok Han (Co-advising with Yongdae Kim)}{Ph.D. in Fall 2022} \\
      First employment: Postdoc at Georgia Tech
    \end{topic}
\end{itemize}

\end{document}
