% Last updated: 5/21/2017

\documentclass[11pt,letterpaper]{article}

\usepackage{enumitem}
\usepackage{parskip}
\usepackage{url}
\usepackage{hyperref}
\usepackage[left=0.75in,top=0.6in,right=0.75in,bottom=0.6in]{geometry}
\usepackage{bold-extra}
\usepackage{ifthen}
\usepackage{relsize}
\usepackage{comment}
\usepackage{textcomp}
\usepackage{lmodern}
\usepackage{etaremune}

\newcommand\email[1]{\href{mailto:#1}{#1}}
\newcommand{\sectiontitle}[1]{\vspace{1em}\textbf{\Large \textsc{#1}}\vspace{0.5em}\hrule}
\newcommand{\subsectiontitle}[3][]{{\bf #2}\ifthenelse{\equal{#1}{}}{}{, #1}
\hfill {#3}}

\newenvironment{topic}[3][]
{
  \subsectiontitle[#1]{#2}{#3}\vspace{-0.5em}
	\begin{itemize}[leftmargin=1.5em]
    \small
		\renewcommand\labelitemi{}
		\itemsep=-0.4em
}
{
	\end{itemize}
}

\newcommand{\entry}[2]{\item {#1 \hfill {#2} \vspace{0em}}}

\newenvironment{subtopic}
{
  \begin{itemize}[label=$\bullet$]
  \vspace{-5px}
}
{
  \end{itemize}
}

\newcommand{\cc}[1]{\mbox{\smaller[0.5]\texttt{#1}}}


\usepackage{kotex}

\begin{document}

% Name, phone, email, address
{\bf\huge Insu Yun} \vspace{1em}\\
\noindent\begin{tabular}[t]{@{}l}
  Assistant Professor \\
  School of Electrical Engineering, \\
  Korea Advanced Institute of Science and Technology (KAIST)
\end{tabular}
\hfill
\begin{tabular}[t]{r@{}}
\\
Email: \email{insuyun@kaist.ac.kr} \\
Web: \href{https://insuyun.github.io}{https://insuyun.github.io} \\
\end{tabular}

%
% Research Interests
%

\sectiontitle{Research Interests}
System security, software security, binary analysis, fuzzing, and applied cryptography.

%
% Education
%

\sectiontitle{Education}
\begin{topic}{Georgia Institute of Technology}{Aug. 2015 -- Dec. 2020}
	\item Ph.D. in Computer Science
	\item Advisor: Dr. Taesoo Kim
\end{topic}

\begin{topic}{Korea Advanced Institute of Science and Technology (KAIST)}{Sep. 2008 -- Feb. 2015}
\item B.S. in Computer Science \& Mathematics
\end{topic}

%
% Work Experience
%

\sectiontitle{Work Experience}
\begin{topic}[Daejeon, South Korea]{KAIST}{Feb. 2021 --}
\item Assistant Professor
\end{topic}
\begin{topic}[Research Intern, Seattle, WA]{Microsoft Research}{May. 2017 -- Aug. 2017}
\item Contributed to REPT, a system that utilizes Intel Processor Trace to diagnose production failures
  \item Mentor:  Weidong Cui
\end{topic}
\subsectiontitle[Research Assistant, Atlanta, GA]{Georgia Tech}{Aug. 2015 -- Dec. 2020} \\
\begin{topic}[Software Developer, Seoul, Korea]{Korean Cyber Command}{Apr. 2012 -- Jan. 2014}
  \item Served for the mandatory military service
\end{topic}


%
% Publications
%

\sectiontitle{Publications}

\textbf{International Journal}
\begin{etaremune}
  {{ INTL_JOURNAL }}
\end{etaremune}

\textbf{International Conferences \forkaist{(\toptier{Top-tier} and others)}{}}
\begin{etaremune}
  {{ INTL_CONF }}
\end{etaremune}

\textbf{Domestic Conferences}
\begin{etaremune}
  {{ DOM_CONF }}
\end{etaremune}

\textbf{Thesis}
\begin{etaremune}
\item \textbf{Concolic Execution Tailored for Hybrid Fuzzing}  \\
{\footnotesize
  \textbf{Insu Yun} \\
  Ph.D. thesis, Georgia Institute of Technology \\
Atlanta, GA, December 2020
}
\end{etaremune}

\pagebreak

\sectiontitle{Professional Activities}
\begin{topic}{Technical Program Committee (International)}{}
  \item{Program Committee, \emph{Network and Distributed System Security Symposium (NDSS)}, 2024}
  \item{Program Committee, \emph{IEEE Symposium on Security and Privacy (Oakland)}, 2024}
  \item{Program Committee, \emph{ACM Conference on Security and Privacy in Wireless and Mobile Networks (WiSec)}, 2023}
  \item{Program Committee, \emph{ACM Conference on Security and Privacy in Wireless and Mobile Networks (WiSec)}, 2022}
\end{topic}

\begin{topic}{Others (International \& Domestic)}{}
  \item{Artifact Evaluation Committee, \emph{USENIX Security Symposium (Security)}, 2023}
  \item{Organization Committee, \emph{ACM Conference on Computer and Communications Security (CCS)}, 2021}
  \item{Organization Committee, \emph{Conference on Information Security and Cryptography Summer (CISC-S)}, 2021}
\end{topic}

% \begin{topic}{International Conference (External Reviewer)}{}
%   \item{External Reviewer, \emph{Network and Distributed System Security Symposium (NDSS)}, 2023}
%   \item{External Reviewer, \emph{Network and Distributed System Security Symposium (NDSS)}, 2022}
%   \item{External Reviewer, \emph{Network and Distributed System Security Symposium (NDSS)}, 2021}
%   \item{External Reviewer, \emph{Network and Distributed System Security Symposium (NDSS)}, 2020}
%   \item{External Reviewer, \emph{ACM Conference on Computer and Communications Security (CCS)}, 2019}
%   \item{External Reviewer, \emph{USENIX Symposium on Operating Systems Design and Implementation (OSDI)}, 2018}
%   \item{External Reviewer, \emph{USENIX Security Symposium (Security)}, 2018}
%   \item{External Reviewer, \emph{USENIX Annual Technical Conference (ATC)}, 2018}
%   \item{External Reviewer, \emph{Network and Distributed System Security Symposium (NDSS)}, 2018}
%   \item{External Reviewer, \emph{ACM Conference on Computer and Communications Security (CCS)}, 2017}
%   \item{External Reviewer, \emph{ACM Conference on Computer and Communications Security (CCS)}, 2015}
%   \item{External Reviewer, \emph{IEEE Symposium on Security and Privacy (Oakland)}, 2015}
%   \item{External Reviewer, \emph{ACM Conference on Computer and Communications Security (CCS)}, 2014}
%   \item{External Reviewer, \emph{ACM Conference on Security and Privacy in Wireless and Mobile Networks (WiSec)}, 2014}
%   \item{External Reviewer, \emph{IEEE Symposium on Security and Privacy (Oakland)}, 2014}
%   \item{External Reviewer, \emph{Network and Distributed System Security Symposium (NDSS)}, 2014}
%   \item{External Reviewer, \emph{IEEE Symposium on Security and Privacy (Oakland)}, 2013}
% \end{topic}

% Teaching Experience
\sectiontitle{Teaching Experience}
% \textbf{Instructor}
\begin{topic}{}{}
  \entry{Software Hacking  Theory and Practice (EE517 at KAIST)}{Spring 2023}
  \begin{subtopic}
    \item Evaluation --  Average: 4.54 / 5
  \end{subtopic}
  \entry{My Life and Career in EE I (EE485-C at KAIST)}{Spring 2023}
  \begin{subtopic}
    \item Evaluation --  Average: 4.80 / 5
  \end{subtopic}
  \entry{Programming Structures for Electronical Engineering (EE209 at KAIST)}{Fall 2022}
  \begin{subtopic}
    \item Evaluation --  Average: 4.65 / 5
  \end{subtopic}
  \entry{Software development environment and tools practice (EE485-A at KAIST)}{Fall 2022}
  \begin{subtopic}
    \item Evaluation --  Average: 4.43 / 5
  \end{subtopic}
  \entry{My Life and Career in EE II (EE485-C at KAIST)}{Fall 2022}
  \begin{subtopic}
    \item Evaluation --  Average: 4.70 / 5
  \end{subtopic}
  \entry{Software Security (EE595-B at KAIST)}{Spring 2022}
  \begin{subtopic}
    \item Evaluation --  Average: 5 / 5
  \end{subtopic}
  \entry{My Life and Career in EE I (EE485-C at KAIST)}{Spring 2022}
  \begin{subtopic}
    \item Evaluation --  Average: 4.65 / 5
  \end{subtopic}
  \entry{Programming Structures for Electronical Engineering (EE209 at KAIST)}{Fall 2021}
  \begin{subtopic}
    \item Evaluation --  Average: 4.34 / 5
  \end{subtopic}
  \entry{Software development environment and tools practice (EE485-A at KAIST)}{Fall 2021}
  \begin{subtopic}
    \item Evaluation --  Average: 4.34 / 5
  \end{subtopic}
  \entry{My Life and Career in EE II (EE485-C at KAIST)}{Fall 2021}
  \begin{subtopic}
    \item Evaluation --  Average: 4.57 / 5
  \end{subtopic}
  \entry{Software Security (EE595-B at KAIST)}{Spring 2021}
  \begin{subtopic}
    \item Evaluation --  Average: 4.9 / 5
  \end{subtopic}
\end{topic}

% \textbf{Teaching Assistant}
% \begin{topic}{}{}
%   \entry{Teaching Assistant, Information Security Lab -- Official (CS8803 at Georgia Tech)}{Fall 2018}
%   \begin{subtopic}
%     \item Evaluation --  Overall Effectiveness: 5 / 5
%   \end{subtopic}
%   \entry{Teaching Assistant, Information Security Lab -- Unofficial (CS8803 at Georgia Tech)}{Fall 2017}
%   \entry{Teaching Assistant, Information Security Lab -- Official (CS6265 at Georgia Tech)}{Fall 2016}
%   \begin{subtopic}
%     \item Evaluation --  Overall Effectiveness: 4.9 / 5
%   \end{subtopic}
%   \entry{Teaching Assistant, Information Security Lab -- Unofficial (CS6265 at Georgia Tech)}{Fall 2015}
%   \entry{Head Instructor, Information Security class for freshmen (KAIST)}{Mar. 2009 -- Aug. 2011}
% \end{topic}

% Honors & Awards
\sectiontitle{Honors \& Awards}
\begin{topic}{Academic awards}{}
  \entry{Best Teaching Award, KAIST Electrical Engineering}{Sep. 2021}
  \entry{Jay Lepreau Best Paper Award, USENIX OSDI 2018}{Aug. 2018}
  \entry{Distinguished Paper Award, USENIX Security 2018}{Aug. 2018}
  \entry{KISA Bug Bounty Program's Hall of Fame}{2013}
\end{topic}

\begin{topic}{\forkaist{Hacking competitions}{Capture the flag}}{}
  \entry{DEFCON 26 CTF, 1st place (Team DEFKOR00T)}{Aug. 2018}
  \entry{DEFCON 24 CTF, 3rd place (Team DEFKOR)}{Aug. 2016}
  \entry{DARPA Cyber Grand Challenge (Team Disekt)}{Aug. 2016}
  \entry{DEFCON 23 CTF, 1st place (Team DEFKOR)}{Aug. 2015}
  \entry{Whitehat contest 2014, 1st place (Team SysSec)}{Nov. 2014}
  \entry{DEFCON 22 CTF, 10th place (Team GoN)}{Aug. 2014}
  \entry{SECCON CTF 2014, 1st place (TOEFL Beginner)}{Feb. 2014}
  \entry{Codegate CTF 2012, 3rd place (Team GoN)}{Apr. 2012}
  \entry{Secuinside CTF, 3rd place (Team GoN)}{Oct. 2011}
  \entry{ISEC CTF, 1st place (Team GoN)}{Sep. 2011}
  \entry{DEFCON 18 CTF, 3rd place (Team GoN)}{Aug. 2010}
  \entry{Codegate CTF 2010, 5th place (Team GoN)}{Apr. 2010}
  \entry{KISA HDCON, Gold Medal, 2nd place (Team GoN)}{May 2009}
  \entry{Codegate CTF 2009, 4th place (Team GoN)}{Apr. 2009}
\end{topic}
%
\begin{topic}{Scholarships}{}
  \entry{National Research Foundation of Korea Scholarship for Undergraduate}{Mar. 2008 -- Dec. 2013}
\end{topic}

\sectiontitle{Vulnerability Discovery Reward (aka Bug bounty)}
To summarize, \$113K bug bounties are awarded so far.

% TODO: Add type confusion
\begin{topic}{With my students}{}
  \entry{Type confusion in V8 (\$7,000), Google, by Haein Lee}{Mar. 2023}
  \entry{NAS authentication bypass in Exynos (\$14,760), Samsung Electronics, by Eunsoo Kim and CheolJun Park}{Feb. 2022}
\end{topic}
%

\begin{topic}{By myself}{}
  \entry{PSV-2021-0304: afpd auth bypass (\$300), NETGEAR}{Mar. 2021}
  \entry{Pwn2Own Apple Safari with a kernel privilege escalation (\$70,000), Zero Day Initiative,
    with Yonghwi Jin and Jungwon Lim}{Mar. 2020}
  \entry{Apple Safari sandbox escape (\$20,000), Apple}{Dec. 2019}
  \entry{Three integer overflow vulnerabilities in PHP (\$1,500), the Internet Bug Bounty}{Jun. 2016}
  \entry{An Integer Overflow in Python zipimport (\$1,000), the Internet Bug Bounty}{Apr. 2016}
\end{topic}


\begin{comment}
\sectiontitle{Reported Security Vulnerabilities}
  {{ CVE }}
\end{comment}

\sectiontitle{Patents}
\newcommand{\patent}[5]{
  \item{\textbf{#1}}{}
  \begin{itemize}[label=]
    \setlength\itemsep{-0.5em}
    \vspace{-5px}
    \item Inventors: #2
    \item Registration date: #3
    \item Patent number: #4
    \item Country: #5
  \end{itemize}
}

\newcommand{\patentpending}[5]{
  \item{\textbf{#1 (Pending)}}{}
  \begin{itemize}[label=]
    \setlength\itemsep{-0.5em}
    \vspace{-5px}
    \item Inventors: #2
    \item Application date: #3
    \item Application number: #4
    \item Country: #5
  \end{itemize}
}

\textbf{International}
\begin{etaremune}
  \patentpending{Security analysis system and method based on negative testing for protocol implementation of LTE device}{Yongdae Kim, Cheoljun Park, Sangwook Bae, Beomseok Oh, Jiho Lee, Mincheol Son, Insu Yun}{2022.10.05}{17960246}{US}
  \patent{Reverse debugging of software failures}{Weidong Cui, Xinyang Ge, Baris Kasikci, Cengiz Can, Ben Niu, Ruoyu Wang, Insu Yun}{10565511}{2020.02.18}{US}
\end{etaremune}

\textbf{Domestic}
\begin{etaremune}
  \patent{LTE 단말기의 프로토콜 구현에 대한 네거티브 테스팅 기반 보안 분석 시스템 및 그 방법}{김용대, 박철준, 배상욱, 오범석, 이지호, 손민철, 윤인수}{10-2514797-0000}{2023.03.23}{Korea}
  \patent{이동통신 표준 문서를 기반으로 이동통신 베이스밴드 소프트웨어를 자동으로 비교 분석하는 방법 및 시스템}{김용대, 김은수, 김동관, 박철준, 윤인수}{10-2546946-0000}{2023.06.20}{Korea}
  \patentpending{키 관리 서비스 제공방법 및 이를 위한 시스템}{한동수, 한주형, 윤인수}{10-2021-0154174}{2021.11.10}{Korea}
  \patentpending{LTE 단말기의 프로토콜 구현에 대한 네거티브 테스팅 기반 보안 분석 시스템}{김용대, 박철준, 배상욱, 오범석, 이지호, 손민철, 윤인수}{10-2021-0133149}{2021.10.07}{Korea}
\end{etaremune}

% Invited Talks
\sectiontitle{Invited Talks}

\newcommand{\talk}[1]{\entry{Title: #1}}

\textbf{International}
\begin{topic}{}{}
  \talk{How to build Skynet --- a system that hacks systems}{}
  \begin{itemize}[label=]
    \setlength\itemsep{-0.5em}
    \vspace{-5px}
    \entry{Keynote speech at TyphoonCon, Seoul, Korea}{Jun. 2023} % 2023.06.15
  \end{itemize}

  \talk{HardsHeap: A Universal and Extensible Framework for Evaluating Secure Allocators}{}
  \begin{itemize}[label=]
    \setlength\itemsep{-0.5em}
    \vspace{-5px}
    \entry{Presented at ACM CCS 2021, Online}{Nov. 2021}
  \end{itemize}

  \talk{Automatic Techniques to Systematically Discover New Heap Exploitation Primitives}{}
  \begin{itemize}[label=]
    \setlength\itemsep{-0.5em}
    \vspace{-5px}
    \entry{Presented at USENIX Security 2020, Online}{Aug. 2020}
  \end{itemize}

  \talk{QSYM: A Practical Concolic Execution Engine Tailored for Hybrid Fuzzing}{}
  \begin{itemize}[label=]
    \setlength\itemsep{-0.5em}
    \vspace{-5px}
    \entry{Presented at USENIX Security 2018, Baltimore, MD}{Aug. 2018}
  \end{itemize}

  \talk{APISan: Sanitizing API Usages through Semantic Cross-checking}{}
  \begin{itemize}[label=]
    \setlength\itemsep{-0.5em}
    \vspace{-5px}
    \entry{Presented at USENIX Security 2016, Austin, TX}{Aug. 2016}
  \end{itemize}
\end{topic}


\textbf{Domestic}
\begin{topic}{}{}
  \talk{Human-friendly binary analysis}{}
  \begin{itemize}[label=]
    \setlength\itemsep{-0.5em}
    \vspace{-5px}
    \entry{Presented at Korea Computer Congress (KCC), Seoul, Korea}{Jun. 2023} % 2023.06.19
  \end{itemize}

  \talk{Exploit in the wild}{}
  \begin{itemize}[label=]
    \setlength\itemsep{-0.5em}
    \vspace{-5px}
    \entry{Presented at ETRI, Daejeon}{Jun. 2023} % 2023.06.14
  \end{itemize}

  \talk{Hacking 101}{}
  \begin{itemize}[label=]
    \setlength\itemsep{-0.5em}
    \vspace{-5px}
    \entry{Presented at WISC, Seoul}{Sep. 2022} % 2022.9.6
  \end{itemize}

  \talk{Attack and Defenses for Heap Vulnerabilities in 2022}{}
  \begin{itemize}[label=]
    \setlength\itemsep{-0.5em}
    \vspace{-5px}
    \entry{Seminar at ETRI, Daejeon}{Apr. 2022} % 2022.4.12
  \end{itemize}

  \talk{Comparative Analysis of Baseband Software and Cellular Specifications for Finding Vulnerabilities}{}
  \begin{itemize}[label=]
    \setlength\itemsep{-0.5em}
    \vspace{-5px}
    \entry{Seminar at UNIST, Ulsan}{May. 2023}  % 2023.5.17
    \entry{Seminar at Security@KAIST, Online}{Jun. 2022} % 2022.06.22
    \entry{Seminar at Cyber Operations Command, Seoul}{Jun. 2022} % 2022.06.10
  \end{itemize}

  \talk{Scalable and Automatic Vulnerability Discovery Beyond Random Testing}{}
  \begin{itemize}[label=]
    \setlength\itemsep{-0.5em}
    \vspace{-5px}
    Seminar at Seoul National University, Seoul, Korea, Mar. 2019 % 2019.03.11
  \end{itemize}

  \talk{Memory Allocator Security}{}
  \begin{itemize}[label=]
    \setlength\itemsep{-0.5em}
    \vspace{-5px}
    \entry{Presented at Best of Best (BoB), Seoul}{Feb. 2023}  % 2023.02.09
    \entry{Presented at Computer System Society Conference (CSC), Pyeongchang}{Feb. 2023}  % 2023.02.07
    \entry{Seminar at UNIST, Online}{May. 2022}  % 2022.5.18
    \entry{Seminar at Yonsei university, Online}{Apr. 2022}  % 2022.4.28
    \entry{Seminar at Sungkyunkwan university, Online}{Apr. 2022}  % 2022.4.27
    \entry{Seminar at ETRI, Daejeon}{Jan. 2022}  % 2022.1.25
    \entry{Seminar at National Security Research Institute (NSRI), Daejeon}{Dec. 2021}  % 2021.12.09
    \entry{Seminar at Securty@KAIST, Online}{Nov. 2021}  % 2021.11.24
    \entry{Seminar at KAIST GSIS, Online}{Nov. 2021} % 2021.11.23
  \end{itemize}

  \talk{Browser Security: Hacking \& Research}{}
  \begin{itemize}[label=]
    \setlength\itemsep{-0.5em}
    \vspace{-5px}
    \entry{Presented at Open Theori Research Seminar \#6, Online}{Dec. 2021} % 2021.12.23
    \entry{Seminar at Hanyang University, Online}{Nov. 2021} % 2021.11.17
    \entry{Presented at KR Becks Meetup \#1 by LINE, Online}{Aug. 2021} % 2021.08.13
    \entry{Seminar at Security@KAIST, Online}{Jun. 2021} % 2021.06.23
\end{itemize}


\end{topic}
\sectiontitle{Grants}
\newcommand{\grant}[5]{
  \entry{\textbf{#1}}{#5}
  \begin{itemize}[label=]
    \setlength\itemsep{-0.5em}
    \vspace{-5px}
    \item Agency/Company: #2
    \item Money: \$#3
    \item Role: #4
  \end{itemize}
}

To summarize, \$1.4 million is awarded, and my portion is \$0.93 million.
Please note that I have accounted for the exchange rate of 1,000 won to one dollar.
\vspace{-10px}

\begin{topic}{}{}
  \grant{Building a system to assist variant analysis for browser}{NRF}{65K}{PI}{23.06 -- 24.05}
  \grant{Developing a security model based on intermediate language in JavaScript}{NSRI}{54.5K}{PI}{23.04 -- 23.10}
  \grant{Studying security threats and safety of open-sourced operating systems}{NSRI}{54.5K}{PI}{23.04 -- 23.10}
  \grant{Automated exploit generation framework for multi-type kernel bugs}{CISC}{100K}{PI}{23.02 -- 23.11}
  \grant{Browser fuzzing using cross-architecture formal verification}{NRF}{110K}{PI}{22.09 -- 23.09}
  \grant{Constructing test cases for validating vulnerability detection}{ETRI}{27.3K}{PI}{22.08 -- 22.11}
  \grant{Developing security model based for testing JavaScript}{NSRI}{54.5K}{PI}{22.04 -- 22.10}
  \grant{Event-driven systems for collecting and analyzing automotive artifacts}{Dankook University Industry-Academia Cooperation Team}{300K $\times$ 0.5}{PI with Prof. Yujip Won}{22.04 -- 23.12}

  \grant{6G security}{Samsung Electronics}{200K $\times$ 0.2}{Co-PI with Prof. Yongdae Kim}{21.08 -- 23.09}
  \grant{DRAM security}{Samsung Electronics}{180K $\times$ 0.2}{Co-PI with Prof. Yongdae Kim}{21.07 -- 24.06}

  \grant{Systematic and precise transformation of the Qualcomm Hexagon architecture into intermediate representations for binary analysis}{NRF}{46.7K}{PI}{21.06 -- 22.05}
  \grant{Automatic generating security models for web browser vulnerability}{NSRI}{54.5K}{PI}{21.04 -- 21.10}
  \grant{Scalable cyber reasoning systems (Start-up)}{KAIST}{150K}{PI}{21.02 -- 24.12}
\end{topic}

\sectiontitle{Advising and Mentoring}
\newcommand{\alumni}[3]{
  \entry{#1, #2}{#3}
}
  \textbf{Ph.D./M.S Students}
  \begin{topic}{}{}
      \entry{- Eunkyu Lee}{Fall 2023}
    \end{topic}

  \textbf{Ph.D./M.S Students}
    \begin{topic}{}{}
      \entry{- Haein Lee}{Spring 2022}
    \end{topic}

  \textbf{M.S. Students}
    \begin{topic}{}{}
      \entry{- Minwoo Baek}{Spring 2022}
      \entry{- Wonyeong Jung}{Spring 2022}
      \entry{- Junyeong Park}{Spring 2022}
      \entry{- Dongok Kim}{Spring 2023}
      \entry{- Seunggi Min}{Fall 2023}
    \end{topic}

  \textbf{Alumni}
    \begin{topic}{}{}
      \alumni{- Hyunsik Jeong (Co-advising with Yongdae Kim)}{S2W}{M.S. in Fall 2021}
      \alumni{- Hyunseok Han (Co-advising with Yongdae Kim)}{Postdoc at Georgia Tech}{Ph.D. in Fall 2022}
    \end{topic}

\end{document}
